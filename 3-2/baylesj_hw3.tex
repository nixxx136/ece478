\documentclass[12pt,letterpaper]{article}

\author{Jordan Bayles}
\title{Homework 3\\
\small ECE 478: Network Security}

%\date{}

%%Usepackage declarations
\usepackage[left=1in,top=1in,right=1in,bottom=1in]{geometry}
\usepackage{lastpage}
\usepackage{sectsty}
\usepackage{slashed}
\usepackage{amsmath}
\usepackage{amsfonts}
\usepackage{latexsym}
% Include for easy import of full pdf pages
\usepackage{pdfpages}
% Include for use of images
\usepackage{graphicx}
% Include for use of [H] placement specifier
\usepackage{float}
% Include for use of \toprule, \midrule, \bottomrule in tabular env.
\usepackage{booktabs}
% Include for setting spacing between lines
\usepackage{setspace}
% Code listing packages
\usepackage{listings}
\usepackage{xcolor}
\usepackage{color}
\usepackage[font=small,format=plain,labelfont=bf,up,textfont=it,up]{caption}
\usepackage[hyphens]{url}

%% Package usages
\sectionfont{\normalsize}
\subsectionfont{\small}

%% New commands
\newcommand{\comment}[1]{}
\newcommand{\field}[1]{\mathbb{#1}} % requires amsfonts
\newcommand{\script}[1]{\mathcal{#1}} % requires amsfonts
\newcommand{\pd}[2]{\frac{\partial#1}{\partial#2}}

%% Access document variables
\makeatletter
\let\thetitle\@title
\let\theauthor\@author
\let\thedate\@date
\makeatother

%% Color Definitions
\definecolor{dkgreen}{rgb}{0,0.6,0}
\definecolor{gray}{rgb}{0.5,0.5,0.5}
\definecolor{mauve}{rgb}{0.58,0,0.82}
\definecolor{lightgrey}{gray}{0.8}
\definecolor{darkgrey}{gray}{1.6}

%% Code Listing Configuration
\DeclareCaptionFormat{listing}{\colorbox{gray}{\parbox{0.987\linewidth}{#1#2#3}}}
\captionsetup[lstlisting]{format=listing, labelfont=white, indention=0pt, textfont=white, margin=0pt, font={bf,footnotesize}, singlelinecheck=false}
\DeclareCaptionFont{white}{\color{white}}
\renewcommand{\lstlistingname}{Code}
\lstset{ %
  %Some lang opts: C++, C, Java, Python, Matlab, TeX, HTML, SQL, Verilog, VHDL, make, ...
  basicstyle=\footnotesize\ttfamily , % the size of the fonts that are used for the code
  numbers=left,                       % where to put the line-numbers
  numberstyle=\scriptsize\color{darkgray}, % the style that is used for the line-numbers
  stepnumber=2,                       % the step between two line-numbers. 
  numbersep=5pt,                      % how far the line-numbers are from the code
  backgroundcolor=\color{white},      % choose the background color. You must add \usepackage{color}
  showspaces=false,                   % show spaces adding particular underscores
  showstringspaces=false,             % underline spaces within strings
  showtabs=false,                     % show tabs within strings adding particular underscores
  frame=tb,                           % adds a frame around the code
  rulesepcolor=\color{gray},          % if not set, the frame-color may be changed on line-breaks within not-black text (e.g. commens (green here))
  tabsize=2,                          % sets default tabsize to 2 spaces
  captionpos=t,                       % sets the caption-position
  breaklines=true,                    % sets automatic line breaking
  breakatwhitespace=false,            % sets if automatic breaks should only happen at whitespace
  title=\lstname,                     % show the filename of files included with \lstinputlisting;
  keywordstyle=\color{blue},          % keyword style
  commentstyle=\color{dkgreen},       % comment style
  stringstyle=\color{mauve},          % string literal style
  escapeinside={\%*}{*)},             % if you want to add a comment within your code
  morekeywords={*,...}                % if you want to add more keywords to the set
  framexbottommargin=5pt,
}

\begin{document}
\begin{flushright}
\theauthor\\
\thedate
\end{flushright}
\begin{center}
\thetitle
\end{center}

\section*{0. Disclaimer}
\emph{This submission reflects my own understanding of the homework and
solutions. All of the ideas are my own, unless I explicitly acknowledge otherwise.}

\section*{1. The secret name of the cookie}
This page sets a cookie upon new account creation called
\textbf{ecbhw}.

\begin{verbatim}
    $cookie = CGI::Cookie->new(
        -name  => "ecbhw",
        -value => encode_base64(encrypt($data))
    );
\end{verbatim}

\section{2. What must a token for the admin user contain?}
The script uses the cookie to check authorization, and splits
up the base64 cookie value into three sections:
\begin{enumerate}
\item ``ok'' - did we decode correctly?
\item ``timestamp'' - account creation time
\item ``username'' - what is the account username?
\end{enumerate}

\begin{verbatim}
if ($cookie && $cookie->value) {
    my $dat = decrypt(decode_base64($cookie->value));
    my ($ok, $timestamp, $username) = split /\|/, $dat;
    if ($ok eq "ok") {
        $AUTHORIZED = $username;
        my $t = localtime($timestamp);
        msg "Welcome back, $username (member since $t)";
    } else {
        msg "Invalid authentication cookie detected";
    }
\end{verbatim}

\section*{3. Manual verification of use of ECB}
You would have to manually modify the cookie and see how the resulting
value is decoded. For example, I created a cookie with username \textbf{metatron}. The login page is thus this:

\begin{verbatim}
Info: Welcome back, metatron (member since Wed Apr 23 07:15:49 2014)
\end{verbatim}

With this cookie currently set to the following content value:

\begin{verbatim}
qLIsoB4lE%2FacKIMVqXvlmw9s%2Fw0cRcV%2B7Oxx8e%2BolY8%3D%0A
\end{verbatim}

We can modify the content in an area that is presumably the username:

\begin{verbatim}
qLIsoB4lE%2FacKIMVqXvlmw6s%2Fw0cRcV%2B7Oxx8e%2BolY8%3D%0A
\end{verbatim}

And it turns out it is bizarrely easy to make modifications to the
cookie without affecting the resulting output, like this example

\begin{verbatim}
qLIsoB4lE%2FacKIMVqXvlmw9s%2Fw0cRcV%2A%2C%2B%2A7Oxx8e%2BolY8
\end{verbatim}

and even easier to get a bad decryption:

\begin{verbatim}
bad decrypt 140286217791304:error:06065064:digital envelope
routines:EVP_DecryptFinal_ex:bad decrypt:evp_enc.c:596:

Info: Welcome back, me (member since Wed Apr 23 07:15:49 2014)
\end{verbatim}

The HTML encoding used to represent the cookie content in this example
makes it extremely difficult to get a working modification.
This example results in a modified user name:

\begin{verbatim}
qLIsoB4lE%2F%2AacKIMVqXvlmw9s%2Fw0cRcV%2F%2C%2B7Oxx8e%2BolY8
\end{verbatim}

Although it really isn't what we are after either.

\begin{verbatim}
bad decrypt 140703959312200:error:0606506D:digital envelope
routines:EVP_DecryptFinal_ex:wrong final block length:evp_enc.c:589:

Info: Welcome back, mežWZQ:nAsŸ‚7ŒŸm6i (member since Wed Apr 23 07:15:49 2014)

\end{verbatim}

On the plus side, this error does tell us the server is using OpenSSL at least.

\section*{4. Block length?}
It is difficult to tell the block length from part 3, but it is likely
a 32 bit block (4 ASCII characters), as the dissimilar portion in my attempts
to solve is 22, meaning it's not a 64 bit block.

\begin{verbatim}
aaaaaaaaaaaaaaabaaaaaaaaaaaaa 
IpdkZbQ21zpl%2BcTuK4LwV04wrzlBaDIzrNZLfdvqatcxR60eGLT8AB2%2Bbqrb%2F3Ay%0A
 9yeCyPYO8sQPrG5XG2m09k4wrzlBaDIzrNZLfdvqatcxR60eGLT8AB2%2Bbqrb%2F3Ay%0A

Dissimilar portion (timestamp + error check)
len(9yeCyPYO8sQPrG5XG2m09k) = 22 characters
Similar portion (username)
4wrzlBaDIzrNZLfdvqatcxR60eGLT8AB2%2Bbqrb%2F3Ay%0A

aaaaaaaaaaaaaaaaaaaaaaaaaaaaa 
m%2FALEoX8P3yeQ7tqDXWGRrTGxF5pBZUDA1Zhe0nGotIxR60eGLT8AB2%2Bbqrb%2F3Ay%0A
\end{verbatim}

Judging by the attempts in section (6), the blocks are a size
that is a factor of both 22 and
\section*{5. What parts can you control?}
You can control the username portion of the token data, as the time
is set on the server side and the authentication portion must
remain unchanged.

\section*{6. Encrypted blocks for the admin token}
You can't have a username start with ``admin,'' however you can
get partial blocks for the admin token, by having a username be
``ad(junk)'' and another username being ``(junk 2)min.''

\begin{verbatim}
admia
KgjljG6nHbNqmd9jZT9DU8an7r7UdMs2U6Jpupi1gb0%3D%0A
admib
kduH7RIaHWtwrLMMZItmLlahAjClu6XQUcqneXZOVD0%3D%0A
admic
91G3ncOvwMGRzaGLWD5BzVx2iqMk%2BvEoKBClkTw5mug%3D%0A
admid
8L9oPg8A2lmvX6lWlzW5bA7XmaKTUgKvcI3WyXmZqDQ%3D%0A

bdmin
nQyk1IaVSnF1uxMndw4wv9d0u2qIbHUsko8Rf9tMues%3D%0A
cdmin
%2FrQUOX77Dt39OhxkwnyD2Nd0u2qIbHUsko8Rf9tMues%3D%0A
ddmin
SdTBw37M%2FJsPjMpWGnKNANd0u2qIbHUsko8Rf9tMues%3D%0A

SIMILAR *dmin:
**********************d0u2qIbHUsko8Rf9tMues%3D%0A
\end{verbatim}

The block length is necessary to know which pieces you need to grab and
set to ensure you modify the blocks you properly need to.

\section*{7. Cookie token use to get an admin token}
You can synthesize parts of the cookie containing partial admin tokens--you are
not allowed to create an account starting with admin--in order to create a single
cookie containing a functioning admin cookie. Because the encryption mode is
set to electronic codebook mode, the plaintext in each block is subject to the
same encryption schema (assuming same key), meaning that the position in the block
chain that each piece is subject to is more important than they are together (such
as username \verb~1admin~, which offsets the encryption and cannot help find the
proper token.

\section*{8. Javascript set cookie}
You can set a cookie in javascript by editing \verb~document.cookie~:

\begin{verbatim}
document.cookie='ecbhw=<new value>'
\end{verbatim}

\section*{9. Secret admin message}

\end{document}
