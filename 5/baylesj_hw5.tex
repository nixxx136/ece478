\documentclass[12pt,letterpaper]{article}

\author{Jordan Bayles}
\title{Homework 5\\
\small ECE 478: Network Security}

%\date{}

%%Usepackage declarations
\usepackage[left=1in,top=1in,right=1in,bottom=1in]{geometry}
\usepackage{lastpage}
\usepackage{sectsty}
\usepackage{slashed}
\usepackage{amsmath}
\usepackage{amsfonts}
\usepackage{latexsym}
% Include for easy import of full pdf pages
\usepackage{pdfpages}
% Include for use of images
\usepackage{graphicx}
% Include for use of [H] placement specifier
\usepackage{float}
% Include for use of \toprule, \midrule, \bottomrule in tabular env.
\usepackage{booktabs}
% Include for setting spacing between lines
\usepackage{setspace}
% Code listing packages
\usepackage{listings}
\usepackage{xcolor}
\usepackage{color}
\usepackage[font=small,format=plain,labelfont=bf,up,textfont=it,up]{caption}

%% Package usages
\sectionfont{\normalsize}
\subsectionfont{\small}

%% New commands
\newcommand{\comment}[1]{}
\newcommand{\field}[1]{\mathbb{#1}} % requires amsfonts
\newcommand{\script}[1]{\mathcal{#1}} % requires amsfonts
\newcommand{\pd}[2]{\frac{\partial#1}{\partial#2}}

%% Access document variables
\makeatletter
\let\thetitle\@title
\let\theauthor\@author
\let\thedate\@date
\makeatother

%% Color Definitions
\definecolor{dkgreen}{rgb}{0,0.6,0}
\definecolor{gray}{rgb}{0.5,0.5,0.5}
\definecolor{mauve}{rgb}{0.58,0,0.82}
\definecolor{lightgrey}{gray}{0.8}
\definecolor{darkgrey}{gray}{1.6}

%% Code Listing Configuration
\DeclareCaptionFormat{listing}{\colorbox{gray}{\parbox{0.987\linewidth}{#1#2#3}}}
\captionsetup[lstlisting]{format=listing, labelfont=white, indention=0pt, textfont=white, margin=0pt, font={bf,footnotesize}, singlelinecheck=false}
\DeclareCaptionFont{white}{\color{white}}
\renewcommand{\lstlistingname}{Code}
\lstset{ %
  %Some lang opts: C++, C, Java, Python, Matlab, TeX, HTML, SQL, Verilog, VHDL, make, ...
  basicstyle=\footnotesize\ttfamily , % the size of the fonts that are used for the code
  numbers=left,                       % where to put the line-numbers
  numberstyle=\scriptsize\color{darkgray}, % the style that is used for the line-numbers
  stepnumber=2,                       % the step between two line-numbers. 
  numbersep=5pt,                      % how far the line-numbers are from the code
  backgroundcolor=\color{white},      % choose the background color. You must add \usepackage{color}
  showspaces=false,                   % show spaces adding particular underscores
  showstringspaces=false,             % underline spaces within strings
  showtabs=false,                     % show tabs within strings adding particular underscores
  frame=tb,                           % adds a frame around the code
  rulesepcolor=\color{gray},          % if not set, the frame-color may be changed on line-breaks within not-black text (e.g. commens (green here))
  tabsize=2,                          % sets default tabsize to 2 spaces
  captionpos=t,                       % sets the caption-position
  breaklines=true,                    % sets automatic line breaking
  breakatwhitespace=false,            % sets if automatic breaks should only happen at whitespace
  title=\lstname,                     % show the filename of files included with \lstinputlisting;
  keywordstyle=\color{blue},          % keyword style
  commentstyle=\color{dkgreen},       % comment style
  stringstyle=\color{mauve},          % string literal style
  escapeinside={\%*}{*)},             % if you want to add a comment within your code
  morekeywords={*,...}                % if you want to add more keywords to the set
  framexbottommargin=5pt,
}

\begin{document}
\begin{flushright}
\theauthor\\
\thedate
\end{flushright}
\begin{center}
\thetitle
\end{center}

\section*{0. Disclaimer}
\emph{This submission reflects my own understanding of the homework and
solutions. All of the ideas are my own, unless I explicitly acknowledge otherwise.}

%%%%%%%%%%%%%%%%%%%%%%%%%%%%%%%%%%%%%%%%%%%%%%%%%%%%%%%%%%%%%%%%%%%%%%%%%%%%%%%
\section{Eavesdropping Tasks}
\subsection{What kind of data is being transmitted in cleartext?}

\subsection{What ports, what protocols?}

\subsection{Can you extract identify any meaningful information from the data?
e.g., if a telnet session is active, what is happening in the sesion? If a file
is being transferred, can you identify the data in the file?}
% Make sure you eavesdrop for at least 30 seconds to make sure you get a
% representative sample of the communication.

\subsection{Is any authentication information being sent over the wire? e.g.,
usernames and passwords. If so, what are they? What usernames and passwords can
you discover?}

% Note: the username and password decoding in ettercap is not perfect--
% how else could you view plain text authentication?

\subsection{Is any communication encrypted? What ports?}

%%%%%%%%%%%%%%%%%%%%%%%%%%%%%%%%%%%%%%%%%%%%%%%%%%%%%%%%%%%%%%%%%%%%%%%%%%%%%%%
\section{Replay Attack against the Stock Ticker}
\subsection{Explain exactly how to execute the attack, including the specific
RPCs you replayed.}

\subsection{Explain how you determined that this strategy would work.}

\subsection{Execute your replay attack and show the results of your attack with
a screen capture, text dump, etc. showing that you are controlling the prices
on the stock ticker.}

%%%%%%%%%%%%%%%%%%%%%%%%%%%%%%%%%%%%%%%%%%%%%%%%%%%%%%%%%%%%%%%%%%%%%%%%%%%%%%%
\section{Insertion Attack}

\subsection{You can change the symbols a viewer of the ticker sees by
intercepting the HTML bound for their browser. Write a filter to change the
symbol FZCO to OWND.}

\subsection{Write a filter to affect the prices a user of the stock ticker sees.
Include your filter sources with your submission materials.}

% Make sure you comment your code (use the # character) to explain
% what the filter does and how.

% Warning! It's hard to test whether your filtering setup is working,
% because ettercap is only filtering traffic to and from alice and bob. If you
% run elinks on eve, or forward a port through eve with ssh tunneling, the
% traffic you see as a viewer will not be modified! You can run tcpdump on eve to
% see the modified packets leaving eve, but only the outgoing packets will be
% modified.

% For these questions, you don't need to write filters, just write short answers
% or pseudocode explaining how you would do it.

\subsection{Given the power of etterfilter and the kinds of traffic on this
network, you can actually make significant changes to a machine or machines
that you're not even logged in to. How?}

\subsection{Of the cleartext protocols in use, can you perform any other dirty
tricks using insertion attacks? The more nasty and clever they are, the better.}

%%%%%%%%%%%%%%%%%%%%%%%%%%%%%%%%%%%%%%%%%%%%%%%%%%%%%%%%%%%%%%%%%%%%%%%%%%%%%%%
\section{MITM vs. Encryption}

\subsection{What configuration elements did you have to change?}
%Copy and paste some of this data into a text file and include it in your
%submission materials.

\subsection{Why doesn't it work to use tcpdump to capture this "decrypted" data?}

\subsection{For this exploit to work, it is necessary for users to blindly "click OK"
without investigating the certificate issues. Why is this necessary?}

\subsection{What is the encrypted data they're hiding?}

\end{document}
